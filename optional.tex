%%  Optional Packages to consider.   These packages are compatible with
%%    ncsuthesis.  

%% -------------------------------------------------------------------------- %%
%% Fancy chapter headings
%%  available options: Sonny, Lenny, Glenn, Conny, Rejne, Bjarne
%\usepackage[Sonny]{fncychap}
%\usepackage[Lenny]{fncychap}
%\usepackage[Glenn]{fncychap}
%\usepackage[Conny]{fncychap}
%\usepackage[Rejne]{fncychap}
%\usepackage[Bjarne]{fncychap}

%% -------------------------------------------------------------------------- %%
%% Chapter headings

\makeatletter
\def\thickhrulefill{\leavevmode \leaders \hrule height 1ex \hfill \kern \z@}
\def\@makechapterhead#1{%
  \vspace*{10\p@}%
  {\parindent \z@ 
        \raggedleft
        \reset@font\Large\bfseries
        \begin{tabular}{c|p{15cm}}
          {\qquad\thechapter{}\  }
          &\quad
          \large #1
        \end{tabular}
        \par\nobreak
    \vskip 100\p@
  }}
\def\@makeschapterhead#1{%
  \vspace*{10\p@}%
  {\parindent \z@ 
        \raggedleft
        \reset@font\Large\bfseries
        \begin{tabular}{cp{15cm}}
          {\qquad\hphantom{\thechapter{}}\  }
          &\quad
          \Large #1
        \end{tabular}
        \par\nobreak
    \vskip 100\p@
  }}


%% -------------------------------------------------------------------------- %%
%% Chapter headings

%\usepackage{titlesec}
%\titleformat{\chapter}[display]
%  {\normalfont\Large\raggedleft}
%  {\MakeUppercase{\chaptertitlename}~%
%    \rlap{ \resizebox{!}{1.5cm}{\thechapter}\rule{6cm}{1.5cm}}}
%  {10pt}{\Large}
%\titlespacing*{\chapter}{0pt}{30pt}{20pt}


%%----------------------------------------------------------------------------%%
%% Hyperref package creates PDF metadata and hyperlinks in Table of Contents
%%  and citations.  Based on feedback from the NCSU thesis editor, 
%%  the links are not visually distinct from normal text (i.e. no change
%%  in color or extra boxes).
\usepackage[
  pdfauthor={Brendan Alexander Harmon},
  pdftitle={Tangible Landscape},
  pdfcreator={pdftex},
  pdfsubject={NC State ETD Thesis},
  pdfkeywords={tangible interfaces, geospatial modeling, geographic information systems, embodied cognition, spatial thinking},
  colorlinks=true,
  linkcolor=black,
  citecolor=black,
  filecolor=black,
  urlcolor=black,
]{hyperref}


%% -------------------------------------------------------------------------- %%
%% Microtype - If you use pdfTeX to compile your thesis, you can use
%%              the microtype package to access advanced typographic
%%              features.  By default, using the microtype package enables
%%              character protrusion (placing glyphs a hair past the right 
%%              margin to make a visually straighter edge)
%%              and font expansion (adjusting font width slightly to get 
%%              more favorable justification).
%%              Using microtype should decrease the number of lines
%%              ending in hyphens.
\usepackage{microtype}


% ------------------------------ Typeface -----------------------------
\usepackage{ifxetex}
\ifxetex
  \usepackage{fontspec}
  \defaultfontfeatures{Ligatures=TeX} % To support LaTeX quoting style
  \setmainfont[Mapping=tex-text, Color=textcolor]{HelveticaNeue}
  %\setmainfont[Mapping=tex-text, Color=textcolor]{Avenir LT Std}
\else
  \usepackage[T1]{fontenc}
  \usepackage[utf8]{inputenc}
  \renewcommand{\familydefault}{\sfdefault}
  \usepackage{helvet}
\fi
\chapterfont{\Large} % \sffamily

%%----------------------------------------------------------------------------%%
%% Fonts 

%% ETD guidelines don't specify the font.  You can enable the fonts
%%  by uncommenting the appropriate lines.  Using the default Computer 
%%  Modern fonts is *not* required.  A few common choices are below.
%%  See http://www.tug.dk/FontCatalogue/ for more options.

%% Serif Fonts -------------------------------------------------
%%  The four serif fonts listed here (Utopia, Palatino, Kerkis,
%%  and Times) all have math support.


%% Utopia
%\usepackage[T1]{fontenc}
%\usepackage[adobe-utopia]{mathdesign}

%% Palatino
%\usepackage[T1]{fontenc}
%\usepackage[sc]{mathpazo}
%\linespread{1.05}

%% Kerkis
%\usepackage[T1]{fontenc}
%\usepackage{kmath,kerkis}

%% Times
%\usepackage[T1]{fontenc}
%\usepackage{mathptmx}


%% Sans serif fonts -------------------------

%\usepackage[scaled]{helvet}  % Helvetica
%\usepackage[scaled]{berasans} % Bera Sans
