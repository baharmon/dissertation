%% ------------------------------ Abstract ---------------------------------- %%
\begin{abstract}

Tangible interfaces enable users to 
kinaesthetically interact with computations. 
%
Theoretically this should embody cognition in 
human-computer interaction 
so that users can 
cognitively grasp digital data with their bodies and
offload challenging cognitive tasks onto their bodies.
%
I have co-designed Tangible Landscape --
a tangible interface 
powered by a geographic information system --
and used it to study how
tangible computing mediates spatial thinking.

Tangible Landscape
couples a malleable, interactive physical model 
of a landscape with a digital model of the landscape 
so that users can 3D sketch ideas 
and immediately analyze 
them with scientific rigor. 
The physical and digital models are linked through 
a real-time cycle of 
physical interaction, 3D scanning, geospatial computation, and projection. 
As users interact with a physical model of a landscape -- 
for example by sculpting the topography -- 
the changes are scanned into 
a
%GRASS GIS, an open source 
geographic information system (GIS), 
for geospatial analysis and simulation 
and the resulting analytics are projected onto the physical model. 
%
% applications
My colleagues and I have developed applications 
for Tangible Landscape including 
stormwater management, flood control, 
landscape management and erosion control, 
trail planning, viewshed analysis, solar analysis, wildfire management, 
disease spread management, invasive species management, 
urban growth, and sea level rise adaption.

% experiment
In a series of terrain modeling experiments
my colleagues and I studied how Tangible Landscape 
mediates spatial thinking and performance. 
We used quantitive and qualitative methods to assess 
participants' spatial performance 
using digital and tangible modeling. 
We used geospatial analytics 
such as geomorphometry and per-cell statistics
supplemented by semi-structured interviews and direct observation
to analyze multidimensional spatial performance. 
% findings
We found that 3D sketching with Tangible Landscape 
improves 3D % multidimensional 
spatial performance
and enables a rapid iterative modeling process. 
%Users can intuitively interact 
%with multidimensional data and scientific models
%with Tangible Landscape. 
\end{abstract}


%% ---------------------------- Copyright page ------------------------------ %%
%% Comment the next line if you don't want the copyright page included.
\makecopyrightpage

%% -------------------------------- Title page ------------------------------ %%
\maketitlepage

%% -------------------------------- Dedication ------------------------------ %%
%\begin{dedication}
% \centering To my parents.
%\end{dedication}

%% -------------------------------- Biography ------------------------------- %%
\begin{biography}
Brendan is a PhD candidate at North Carolina State University with
a Bachelor of Arts from Sewanee, the University of the South,
a Master of Landscape Architecture from the Harvard Graduate School of Design,
and
a Master of Philosophy in Geography and the Environment from the University of Oxford.
\end{biography}

% ADD A PICTURE

%% ----------------------------- Acknowledgements --------------------------- %%
%\begin{acknowledgements}
%I would like to thank my advisor for his help.
%\end{acknowledgements}


\thesistableofcontents

\thesislistoftables

\thesislistoffigures
