\chapter{Introduction}
\label{chap-one}

\section{Tangible interaction}

In a seminal paper Ishii and Ullmer 
envisioned tangible user interfaces that would 
`bridge the gap between cyberspace and the physical environment 
by making digital information (bits) tangible' \cite{Ishii1997}.
They described `tangible bits' as 
`the coupling of bits with graspable physical objects' \cite{Ishii1997}. 
Tangible interfaces 
-- interfaces that couple physical and digital data \cite{Dourish2001} -- 
are designed to physically manifest digital data 
so that users can cognitively grasp and absorb it,
so that we can think with it rather than about it \cite{Kirsh2013}. 
%With a tangible interface both input and output are physical;
%this enables users to
%`take advantage of natural physical affordances 
%to achieve a heightened legibility and seamlessness of interaction 
%between people and information' \cite{Ishii1997}. 
Since 
`the physical state of tangibles 
embodies key aspects of the digital state 
of an underlying computation' \citep{Ishii2008}
data is not just visualized, 
but can also be felt and directly, physically manipulated, 
leveraging highly developed motor skills. 
When interactions are embodied 
intention, action, and feedback 
should be seamlessly connected.
Interaction should be automatic, subconscious, and intuitive.

\section{Tangible geospatial interaction}

Scientific computing can used to study and 
solve complex spatial problems. 
It can be challenging, however, 
to visualize and interact with complex spatial computations.
Tangible interfaces for geospatial modeling 
give spatial computations 
an interactive, physical form
that can be kinaesthetically 
explored and transformed.
Users should be able to 
intuitively grasp, understand, and manipulate 
spatial computations.
These tangible interfaces
should computationally augment users' spatial cognition, 
while offloading the cognitive tasks of 
interaction and perception onto their bodies. 

\subsection{Theory}
%
Theoretically tangible interfaces for geospatial modeling
should enable 
intuitive digital sculpting, 
streamline design processes by seamlessly integrating 
analog and digital workflows, and 
improve collaboration and communication by enabling multiple users 
to simultaneously interact 
in a natural way \cite{Ratti2004}. 
GIS provide the extensive libraries for geospatial databasing, 
processing, analysis, modeling, simulation, and visualization 
needed to address real world problems \cite{Tateosian2010}. 
%
Tangible interfaces for geospatial modeling 
have the potential to make spatial science more intuitive 
by enabling a rapid iterative process 
of observation, hypothesizing, testing, and inference.
With these systems users can for example 
dynamically explore how topographic form 
influences landscape processes \cite{Mitasova2006}. 

% review
A review of tangible interfaces for geospatial modeling
(Tables~\ref{tt-tab-1}-\ref{tt-tab-4}), however, 
reveals a dearth of critical research.
While many prototypes have been developed
there have been 
few case studies or qualitative user studies
and no quantitative studies.
The theories that have inspired 
 tangible interfaces for geospatial modeling
have not yet been critiqued or proven. 

\subsection{Prototypes}

Tangible interfaces like 
Urp \cite{Underkoffler1999}, 
Illuminating Clay \cite{Piper2002a}, 
and SandScape \cite{Ratti2004} 
enriched physical models of urban spaces and landscapes 
with spatial analyses and simulations 
like wind direction, cast shadow, 
slope, aspect, curvature, and water direction
in order to enhance and streamline design processes. 
Many of the analyses used in Illuminating Clay 
were adapted from GRASS GIS \cite{Piper2002a}
and eventually 
Illuminating Clay was coupled with GRASS GIS 
to draw on its extensive libraries for spatial computation. 
The aim of coupling Illuminating Clay with GRASS GIS was to 
`explore relationships that occur between different terrains, 
the physical parameters of terrains, 
and the landscape processes 
that occur in these terrains' \cite{Mitasova2006}. 
The research to couple a physical landscape model with GRASS GIS 
led to the development of 
the Tangible Geospatial Modeling System \cite{Tateosian2010} 
and Tangible Landscape \cite{Petrasova2014,Petrasova2015}.
Tangible Landscape couples a malleable physical model 
and a digital geospatial model of a landscape 
through a real-time cycle of physical interaction, 3D sensing, 
geospatial computation, and projection. 
Tangible Landscape was designed for
natural and collaborative interaction with GIS.
Such natural interaction with geospatial models should 
enhance spatial cognition and encourage spatial learning. 

\section{Research}

%\subsection{Aims and objectives}

% aim
The aim of this research was to make
scientific spatial computing 
natural, rapid, and collaborative
% motivation
in order to
streamline spatial science and design. 
% objective
The objective was to 
design a tangible interface for geospatial modeling
and study how it mediates spatial cognition
in order 
to implement, study, critique, and refine 
theories about tangibles and spatial thinking. 
% questions
Through an iterative cycle of design and research 
my colleagues and I addressed the following research questions: 
\begin{itemize}
\item How can we design an effective tangible interface for geospatial modeling?
\item How can we use tangible geospatial analytics to solve spatial problems?
\item How does coupling physical and digital models of topography mediate spatial thinking?
\item How do different tangible geospatial analytics mediate spatial thinking?
\item How do tangible geospatial analytics
mediate how users relate form and process?
\end{itemize}

\subsection{Methods}
%In an iterative cycle of design and research
I co-designed Tangible Landscape 
and ran a series of terrain modeling experiments 
studying how tangible geospatial modeling 
mediates spatial cognition and performance.
The experiments studied how 
coupling physical and digital models
and providing real-time analytic feedback
affected 3D spatial performance.
Spatial performance was assessed using
qualitative methods 
including semi-structured interviews and direct observation
and quantitive methods 
including geospatial modeling, 
analysis, simulation, and statistics.
This research has informed 
the continued design and development
of Tangible Landscape.

\subsection{Results}
% what is new?
Through this research we developed
 a unique tangible interface for geospatial modeling
-- the only real-time system 
powered by a GIS
and capable of a wide variety of 
tangible geospatial models, analyses, and simulations 
\cite{Petrasova2014,Petrasova2015,Harmon2016c}. 
%
We implemented, demonstrated, and tested many novel applications 
for tangible geospatial modeling 
\cite{Petrasova2014,Petrasova2015,Harmon2016c,Petrasova2016}. 
%
% findings
We also conducted the first quantitative, spatial studies 
of tangible geospatial modeling.
%
We found that 
coupled digital and physical models 
can improve 3D modeling performance
\cite{Harmon2016b,Harmon2016c}.
%
We also found that tangible geospatial analytics 
like differencing and water flow
can enable iterative modeling processes \cite{Harmon2016c}
and help users connect form to processes
\cite{Harmon2016,Harmon2016c}.


\section{Organization}

This thesis is a collection of papers and a book about Tangible Landscape.

Chapter \ref{chap-two} 
describes a pilot study for the first terrain modeling experiment 
focused on coupling digital and physical models.
This chapter is a reprint of the paper 
\emph{Embodied Spatial Thinking in Tangible Computing}
from the Proceedings of the TEI '16: 
Tenth International Conference on Tangible, Embedded, and Embodied Interaction
\cite{Harmon2016b}.

Chapter \ref{chap-three} 
describes a pilot study for the third terrain modeling experiment
using Tangible Landscape's water flow analytic. 
This chapter is a reprint of the paper 
\emph{Tangible Landscape: Cognitively Grasping the Flow of Water}
from the International Archives of the Photogrammetry, Remote Sensing and Spatial Information Sciences
\cite{Harmon2016}

Chapter \ref{chap-four}
describes the design of Tangible Landscape 
and then describes the series of terrain modeling experiments. 
This chapter is a draft of the paper 
\emph{Cognitively Grasping Topography with Tangible Landscape},
which has been submitted for publication \cite{Harmon2016c}.

Chapter \ref{chap-five} describes the design of Tangible Landscape, 
its applications, how to set it up, and how use it.
This chapter is a reprint of the book
\emph{Tangible Modeling with Open Source GIS}
\cite{Petrasova2015}.

Appendix \ref{app-a} 
describes the first generation of Tangible Landscape
and demonstrates applications.
This appendix is a reprint of the paper
\emph{GIS-based environmental modeling with tangible interaction and dynamic visualization}
published in the 
Proceedings of the 7th International Congress on Environmental Modelling and Software
 \cite{Petrasova2014}. 
 % Connection to Tangible Landscape
 % Conference presentation

Appendix \ref{app-b}
describes 
how free, open source software can be integrated into 
geospatial education
to promote open and reproducible science. 
Tangible Landscape is used in a couple the case studies 
demonstrating this approach.
This appendix is a reprint of the paper
\emph{Integrating Free and Open Source Solutions into Geospatial Science Education}
published in the
ISPRS International Journal of Geo-Information
\cite{Petras2015}.

Appendix \ref{app-c}
describes the integration of 
the
FUTURES  
(FUTure Urban -- Regional Environment Simulation)
model
into GRASS GIS 
and makes a case for 
GRASS GIS as a platform for 
free, open source geospatial software development. 
This appendix is a reprint of the paper
an \emph{Open source approach to urban growth simulation}
published in
The International Archives of the Photogrammetry, Remote Sensing and Spatial Information Sciences
\cite{Petrasova2016}.

Appendix \ref{app-d}
describes the coupling of Tangible Landscape with an immersive virtual environment 
so that users can visualize and design landscapes at a human-scale. 
This appendix is a preprint 
of the paper
\emph{Immersive Tangible Geospatial Modeling}
that has been accepted for publication
%published in...
in the Proceedings of the 24th ACM SIGSPATIAL 
International Conference on Advances in Geographic Information Systems
\cite{Tabrizian2016}.

Appendix \ref{app-e}
links conference talks 
related to Tangible Landscape
\cite{Harmon2014a,Harmon2016tei,Harmon2016d,Harmon2016e,Harmon2016f,Harmon2016g,Harmon2016h}.
%
%Talks include 
%%
%\emph{Tangible geospatial modeling for landscape architects} 
%at the
%Geodesign Summit 2014
%\cite{Harmon2014a},
%%
%\emph{Embodied Spatial Thinking in Tangible Computing} 
%at
%Tangible, Embedded, and Embodied Interaction (TEI) 2016
%\cite{Harmon2016tei},
%%
%\emph{Serious Gaming with Tangible Landscape} 
%for NCSU's 
%Coffee \& Viz series
%\cite{Harmon2016d},
%%
%\emph{Creative spatial thinking with Tangible Landscape}
%at the
%American Association of Geographers (AAG) Annual Meeting 2016
%\cite{Harmon2016e},
%%
%\emph{Tangible interaction for GIS}
%at
%FOSS4G NA 2016
%\cite{Harmon2016f},
%%
%\emph{Tangible Landscape: cognitively grasping the flow of water}
%at the
%International Society of Photogrammetry and Remote Sensing (ISPRS) 2016
%\cite{Harmon2016g}, and
%%
%Tangible geographies
%at the
%Royal Geographical Society (RGS) Annual International Conference 2016
%\cite{Harmon2016h}.

%Appendix \ref{app-e}
% TL + landscape evolution
% \cite{Harmon2016d}


%Appendix \ref{appendix:tonini2016}
% TL + SOD
% \cite{Tonini2016}






\printbibliography[heading=myheading]