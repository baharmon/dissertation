\documentclass{article}
\usepackage{graphicx}
\graphicspath{{./images/}}
\usepackage{geometry}
\usepackage{hyperref}
\usepackage{paralist}
\usepackage[round]{natbib}
\usepackage{sectsty}
\usepackage{gensymb}
\usepackage{caption}
\usepackage{subcaption}
\usepackage{listings}
\usepackage[space]{grffile}
\usepackage{latexsym}
\usepackage{amsfonts,amsmath,amssymb}
\usepackage{url}
%\usepackage{minitoc}
\hypersetup{colorlinks=false,pdfborder={0 0 0}}
\usepackage{textcomp}
\usepackage{longtable}
\usepackage{multirow,booktabs}
\newcommand{\truncateit}[1]{\truncate{0.8\textwidth}{#1}}
\newcommand{\scititle}[1]{\title[\truncateit{#1}]{#1}} 
\usepackage[parfill]{parskip}

% Typeface
\usepackage{ifxetex}
\ifxetex
  \usepackage{fontspec}
  \defaultfontfeatures{Ligatures=TeX} % To support LaTeX quoting style
  \setmainfont[Mapping=tex-text, Color=textcolor]{HelveticaNeue}
  %\setmainfont[Mapping=tex-text, Color=textcolor]{Avenir LT Std}
\else
  \usepackage[T1]{fontenc}
  \usepackage[utf8]{inputenc}
  \renewcommand{\familydefault}{\sfdefault}
  \usepackage{helvet}
\fi
%\chapterfont{\Large} % \sffamily
%\renewcommand{\chaptername}{}
%\renewcommand{\thechapter}{}

% Title page
\usepackage{xcolor}
\definecolor{titlepagecolor}{cmyk}{0,0,0,0}
\definecolor{namecolor}{cmyk}{0,0,0,1} 
\definecolor{chaptertitlepagecolor}{cmyk}{0,0,0,0.9}
\definecolor{chapternamecolor}{cmyk}{0,0,0,0.3}  

\begin{document}

%---------------------------------------------- BODY ----------------------------------------------

\section{Creativity}

\paragraph{What is creativity?}
% What is creativity
Creativity -- the potential for making something new -- 
is framed by the old, 
is contingent upon history.
As what was once new becomes old and 
as new challenges, new skills, new technologies, and new possibilities emerge 
the potential for making something new evolves.
%
Creativity is situated in time, in cultural and social context. 
It is perceived at temporal scales from the instantaneous to the evolutionary,
at scales ranging from the personal -- the interior -- to the global. 
%
Creativity depends upon the perception or experience of newness, 
a concept that is transient, dependent upon the past, and culturally defined and yet personally determined.
Thus creativity is unstable, continually shifting, continually evolving, 
its possibility, its value, and its definition in constant flux 
yet constrained by history and context.
Creativity allow us to adapt cognitively and culturally
as our world changes.


Art -- as creative expression -- is the evocation of cognitive change \citep{Donald2006}.
Art as an inherently creative endeavor
highlights the inchoate nature of creativity.
`The concept of art is located in a historically changing constellation of elements; it refuses definition'
\citep{Adorno1997}.
Art's definitions and values shift over time and across cultures -- 
`much that was not art -- cultic works, for instance -- has over the course of history metamorphosed into art; 
and much that was once art is that no longer'
\citep{Adorno1997}.
Because art is creative, 
`art can be understood only by its laws of movement.... It is defined by its relation to what it is not.'
\citep{Adorno1997}.
%
%Perhaps creativity should be studied not as a process, 
%but rather generatively and evolutionarily
%as a stream of processes in flux.

%\subsection{Deconstructing creativity}
\paragraph{Charismatic creativity}
% The normative conception of creativity
Creativity does not seem so mysterious. % but rather romantic
%
Psychologists have classified types of creativity, modeled the creative process, 
and developed metrics for measuring creativity.
Creativity has been classified in terms of implementation, novelty, and value, and intentionality. 
Creativity can be personal, novel and significant for the individual, or it can be novel and valued in a wider context, potentially at a global and historical scale \citep{Drago2014}. 
%
\citeauthor{Wallas1926} modeled the creative process as a sequence of stages -- 
a process of preparation, incubation, illumination, and verification \citeyearpar{Wallas1926}. 
In this widely accepted model of creativity 
one first prepares -- 
one begins to think about a problem or endeavor, 
researching it and developing the skills needed to address it.
Eventually one lets their ideas incubate in their subconscious, 
letting their imagination make novel connections. 
When a creative idea emerges from the subconscious 
there is a moment of illumination.
Finally the creative idea should be expressed, tested, and verified.
Does it solve the problem? Is it valuable? Is it truly novel? \citep{Wallas1926, Drago2014}. 
%

These models of charismatic creativity are based on a modern conception of self
and Romantic ideologies of natural genius, originality, and individualism. 
In Romanticism originality is unlearned and inspired from within. 
%
Since originality is unlearned, 
the individual -- not their society -- is paramount; 
this spirit of individualism 
fueled a culture of innovation focused on genius. 
%
Romantic ideas of self expression stressed 
contingency, spontaneity, and unpredictability
in creative work \citep{Wilf2013a,Wilf2013b}. 
%

% Streamlining the muse
%
\citeauthor{Wilf2013b}
argues that with the rise of a `creative class' in contemporary society 
there has been a move to professionalize charismatic creativity by streamlining the muse. 
That the Romantic vision of creativity as an unpredictable, unteachable process that springs from within 
has become ingrained in global culture today, but is now being transformed. 
That our culture seeks to harness contingency and spontaneity as cultural resources 
in order to make creativity more efficient \citeyearpar{Wilf2013b}. 
%


\paragraph{Flow}
% other models of creativity
There are other important psychological and cognitive models of creativity. 
%
% flow
\citeauthor{Csikszentmihalyi2014a} proposed that `flow,' a state of `intense experiential involvement in moment-to-moment activity' in which action and awareness merge \citeyearpar{Csikszentmihalyi2014a} was the primary motivating force, the driver of creativity \citep{Csikszentmihalyi2014}. 
While still based on the romantic vision of charismatic creativity
flow highlights the importance of action, experience, and emotion. 
By grounding creativity in the body 
flow suggests the potential role of embodied cognition 
in creativity. 
And if embodied cognition plays a role in creativity then surely situated and distributed cognition must as well?
%

\paragraph{Conceptual blending}
Conceptual blending -- another model for creativity --
describes the subconscious cognitive process by which meaning evolves.
This model explores creativity at a fine temporal scale -- the instantaneous -- and how it unfolds iteratively.
In a rapid, subconscious process, 
two ideas are blended together against a background of shared knowledge so that a novel idea emerges. 
The two ideas may be mismatched and even contradictory so long as they share something in common.  
The relationship drawn between these two potentially incongruous ideas creates new meaning.
Types of meaning such as a  ``categorizations, analogies, counterfactuals, metaphors, rituals, logical framing, and grammatical
constructions'' are formed by blending \citep{Fauconnier2000}. 
The process can unfold rapidly, sequentially, and in parallel as blends are blended. 
In blends of blends concepts can become more complex and nested.
Greater novelty and richer meaning can arise as more blends are blended. 
This is a very different conception of creativity.
It is a rapid process. It is situated and potentially distributed and embodied. 
In conceptual blending cognition is situated in a background of shared knowledge and experience. 
It may be embodied, drawn from kinaesthetic and emotion experience and expressed in action.
It is distributed across existing ideas, shared meanings. 
It may be distributed between multiple actors sharing creative agency, relying on tools and technology. 

\paragraph{Embodied creativity}
%\paragraph{The evolution of creativity}
Art and thus creativity evolved with cognition.
As early human's prefontal and premotor cortex grew in size and connectivity 
their metacognitive abilities expanded 
so that they could reflect on, develop, and eventually share their actions.
%
Roughly 2 million years ago the first mode of artistic expression, the mimetic arts
 -- `gesturing, pantomime, dance, visual analogy, and ritual' -- emerged
marking a significant advance in cognition, culture, and creativity \citep{Donald2006}. 
Mimesis enabled people to share ideas, technologies, and practices through imitation, rehearsal, and play and set the stage for the emergence of language \citep{Donald2006}. 
%
Mimesis enabled people to embody cognitive processes through action -- 
abstracting ideas through play \citep{Deacon2006} and physically simulation \citep{Kirsh2013}
in order to learn them and adapt them. 
Adaptation is always creative because 
every body is unique,
mimesis is exploratory,
and abstraction introduces novelty. 
%
With mimesis humans had the creative ability to abstractly explore and develop ideas with the body. 
Later when linguistic and then theoretical expression evolved, new modes of creativity and art emerged \citep{Donald2006}. 

%\paragraph{Embodied creativity}
In embodied cognition thinking is embedded in the body. Meaning is grounded in emotion, perception, and experience.
Higher thought is built upon perception, sensation, and action. Thinking can happen through action \citep{Hardy-Vallee2008}. 
Creativity, when embodied, should be conceived differently. 
Embodied cognition relies upon sensory and motor schemas, on parallel subconscious processing. 
Complex thought processes can be offloaded and physically simulated with the body, processed subconsciously. 
Because so much cognitive work can be performed subconsciously through action, 
subconscious creative processes need not be separated into \citeauthor{Wallas1926}'s incubation stage -- 
the creative process may unfold in rapid, generative iterations in which conscious thought is built upon subconscious thinking-in-action.

% other cultures
\paragraph{Creativity in action}
While psychological research has been focused on charismatic creativity
there are other cultures with very different, more embodied conceptions of creativity.
In the Zen Arts, for example, 
`creativity is represented as a state of `nothingness' (m{\^u}shin) that is beyond matter and form ' \citep{Cox2011}, a state of purely embodied, subconscious thinking-in-action.
%
When creativity is embodied there may be no time for incubation, no discrete moment of inspiration or illumination, no linear step-by-step process. 
%
While creative performances may (or may not) be choreographed and practiced, the performers make creative decisions while in action. Dancers may interpret their emotions through dance, actors interpret their roles while acting them, and martial artists creatively act and react. 
%
Open work -- art in which the author, the performer, and potentially even the audience collaborate and develop meaning together -- is a radical example of creative agency in performance.
In Stockhausen's Klavierst{\"u}ck XI for example the performers decide 
how to navigate through a sequence of fragments, 
charting their own course across the sheet of music in a highly creative performance that unfolds in action \citep{Eco1989}. 
%

\paragraph{Deconstructing creativity}
The role of the individual creative -- the author, the artist, the genius -- as the sole creative agent, as the fount of meaning
has been deconstructed by \citeauthor{Barthes1977} \citeyearpar{Barthes1977}  and \citeauthor{Foucault1998} \citeyearpar{Foucault1998}. 
%
Every creative work in situated in an intertextual network; 
it was inspired by other works, references others, communicates through them.
A work's meaning is not transmitted directly from the creative to their audience; it is filtered by the audience's awareness of other texts, other works, other experiences that informed it. Creative agency unravels, distributed. 
The reader interprets the work, creating their own meaning.  
The network of texts mediates, contributes possible meanings, other readings.

\paragraph{Creativity in design}
% Creativity in design

The creative process in design disciplines is often described as an iterative process 
in contrast to the linear model espoused by \citeauthor{Wallas1926}. 
%
\citeauthor{Schon1983} proposed that 
professionals like designers develop creative ideas through `reflection-in-action,' 
an iterative, exploratory process 
of framing the problem, ideation or making, and critical reflection \citeyearpar{Schon1983}.
This exploratory process may unfold in an instant in action, 
repeated continually through acts like drawing or model making. 
%
Frank Gehry for example develops his designs through exploratory form finding with massing models
and by thinking through the movement of gestural drawing
\cite{Gehry2004,Pollack2006}.
%

%
The iterative nature of `reflection-in-action' 
may suggest incremental creative advances, but 
novelty and efficacy may emerge suddenly
as the sophistication and complexity of the idea develops. 
The choice of (or process of choosing) the frame -- a generative metaphor perhaps -- is an important creative decision.
%
The frame, however, is not static -- it is developed, refined, adapted, and perhaps even discarded and replaced through the creative process.
Since conceptual blending describes how frames such as metaphors and analogies 
are developed, adapted, and built upon, 
it may be the cognitive process underlying `reflection-in-action.' 
%

\citeauthor{Yaneva2009}'s ethnography of the Office of Metropolitan Architecture
showed that architects in the firm used practices like reuse, adaption, and exploratory modeling
to develop a design through an iterative process lacking \citeauthor{Wallas1926}'s sudden moment of illumination \citeyearpar{Yaneva2009}. 
The architects, for example, explored form by carving foam massing models with hot-wire cutters, reflecting on each model as they carved, while building up a library of forms. Ideas and forms set aside and unused in one project might show up in other projects, recycled and adapted \citep{Yaneva2009}.
%
In design there is a tension between charismatic creativity and collaborative studio work. 
%
While the charismatic `starchitect' Rem Koolhaas dominates the public's perception of the Office of Metropolitan Architecture, 
\citeauthor{Yaneva2009}'s ethnography revealed a collaborative studio environment with distributed creative agency \citeyearpar{Yaneva2009}. 

\paragraph{Assessing creativity}
% Assessing creativity

Just as there are many models of creativity
there are many different methods for assessing creativity. 
%
There are methods for assessing an individual's creativity 
such as psychometric tests \citep{Drago2014}, 
self assessments, peer assessment, and creative output \cite{Csikszentmihalyi2014}. 
%
There are qualitative methods for assessing the results of the creative process 
such as expert judgment and evaluation. 
And there are quantitative experiments for assessing the cognitive response of subjects to a creative product 
(for example a painting or sculpture) 
using technologies like eye tracking, electroencephalograms, and functional magnetic resonance imaging
\cite{Chatterjee2014}.
%
There are qualitative methods for assessing the performance of creativity like
the experience sampling method -- randomized self reporting -- used to assess flow \cite{Csikszentmihalyi2014a}.
%
Since there are many modes of creativity, 
I argue that ethnographic methods such first be used to frame the creative practice
in its cultural context. 
The way in which creativity is conceived in a given context must be established before it can be assessed.
%
Once we have described and framed creative practice in context
we may be able to experimentally study the creative process and its results. 







%---------------------------------------------- BIBLIOGRAPHY ----------------------------------------------

\bibliographystyle{plainnat}
\bibliography{../tangible_topography} 
\end{document}








