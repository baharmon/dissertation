\documentclass{article}
\usepackage{graphicx}
\graphicspath{{./images/}}
\usepackage{geometry}
\usepackage{hyperref}
\usepackage{paralist}
\usepackage[round]{natbib}
\usepackage{sectsty}
\usepackage{gensymb}
\usepackage{caption}
\usepackage{subcaption}
\usepackage{listings}
\usepackage[space]{grffile}
\usepackage{latexsym}
\usepackage{amsfonts,amsmath,amssymb}
\usepackage{url}
%\usepackage{minitoc}
\hypersetup{colorlinks=false,pdfborder={0 0 0}}
\usepackage{textcomp}
\usepackage{longtable}
\usepackage{multirow,booktabs}
\newcommand{\truncateit}[1]{\truncate{0.8\textwidth}{#1}}
\newcommand{\scititle}[1]{\title[\truncateit{#1}]{#1}} 
\usepackage[parfill]{parskip}

% Typeface
\usepackage{ifxetex}
\ifxetex
  \usepackage{fontspec}
  \defaultfontfeatures{Ligatures=TeX} % To support LaTeX quoting style
  \setmainfont[Mapping=tex-text, Color=textcolor]{HelveticaNeue}
  %\setmainfont[Mapping=tex-text, Color=textcolor]{Avenir LT Std}
\else
  \usepackage[T1]{fontenc}
  \usepackage[utf8]{inputenc}
  \renewcommand{\familydefault}{\sfdefault}
  \usepackage{helvet}
\fi
%\chapterfont{\Large} % \sffamily
%\renewcommand{\chaptername}{}
%\renewcommand{\thechapter}{}

% Title page
\usepackage{xcolor}
\definecolor{titlepagecolor}{cmyk}{0,0,0,0}
\definecolor{namecolor}{cmyk}{0,0,0,1} 
\definecolor{chaptertitlepagecolor}{cmyk}{0,0,0,0.9}
\definecolor{chapternamecolor}{cmyk}{0,0,0,0.3}  

\begin{document}

%---------------------------------------------- BODY ----------------------------------------------

\section{Tangible interfaces for Geographic Information Systems}

% You introduce Tangible landscape with "imagine holding GIS in your hands" statement. Elaborate on the coupling of tangible interface with GIS: what role GIS plays to support tangible interaction, what GIS functionality is needed to detect the tangible input and what provides the realtime feedback. Discuss the current bottlenecks and challenges that need to be addressed to fulfill the Tangible Landscape potential and propose innovative solutions.



% Imagine being able to hold a GIS in your hands, feeling the shape of the earth, sculpting its topography, and directing the flow of water. We present Tangible Landscape, an open source tangible interface powered by GRASS GIS. Tangible Landscape physically, interactively manifests geospatial data so that you can naturally feel it, see it, and shape it. This makes GIS far more intuitive and accessible for beginners, empowers geospatial experts, and creates new exciting opportunities for developers - like gaming with GIS. In this talk we will introduce tangible interaction and why it matters for all things spatial, demonstrate a few applications such as disaster management and gaming, discuss how to digitally fabricate models, and show you how to implement and build your own system. 




% coupling



% coupling with GIS


	% detection of tangible input with GIS

	% what sort of GIS feedback





\paragraph{Challenges}
% challenges
	% theory unproven
	% 2 way tangible interaction
	% point of view: human perception of the landscape

Challenges to the development of Tangible Landscape...

% theory unproven
Theoretically Tangible Landscape should enable natural and collaborative interaction with a GIS by coupling a physical and digital geospatial models.
Natural interaction with geospatial models should enhance spatial cognition and encourage spatial learning. 
This should be empirically tested in experiments and case studies
so that we can critique and develop the theory grounding Tangible Landscape, 
identify cognitive challenges, and 
improve the design.  

% 2 way tangible interaction









%point of view: human perception of the landscape








\paragraph{Future work}
% innovations 
	% empirically, experimentally testing theory in order to improve the design
		% spatial cognition and other cognitive, affective, metacognitive, and motivational (CAMM) processes
	% digital fabrication, live data, and earthmoving
	% VR
























%---------------------------------------------- BIBLIOGRAPHY ----------------------------------------------

\bibliographystyle{plainnat}
\bibliography{../tangible_topography} 
\end{document}








